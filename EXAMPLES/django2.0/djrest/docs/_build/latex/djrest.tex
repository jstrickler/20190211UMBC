%% Generated by Sphinx.
\def\sphinxdocclass{report}
\documentclass[letterpaper,10pt,english]{sphinxmanual}
\ifdefined\pdfpxdimen
   \let\sphinxpxdimen\pdfpxdimen\else\newdimen\sphinxpxdimen
\fi \sphinxpxdimen=.75bp\relax

\PassOptionsToPackage{warn}{textcomp}
\usepackage[utf8]{inputenc}
\ifdefined\DeclareUnicodeCharacter
% support both utf8 and utf8x syntaxes
\edef\sphinxdqmaybe{\ifdefined\DeclareUnicodeCharacterAsOptional\string"\fi}
  \DeclareUnicodeCharacter{\sphinxdqmaybe00A0}{\nobreakspace}
  \DeclareUnicodeCharacter{\sphinxdqmaybe2500}{\sphinxunichar{2500}}
  \DeclareUnicodeCharacter{\sphinxdqmaybe2502}{\sphinxunichar{2502}}
  \DeclareUnicodeCharacter{\sphinxdqmaybe2514}{\sphinxunichar{2514}}
  \DeclareUnicodeCharacter{\sphinxdqmaybe251C}{\sphinxunichar{251C}}
  \DeclareUnicodeCharacter{\sphinxdqmaybe2572}{\textbackslash}
\fi
\usepackage{cmap}
\usepackage[T1]{fontenc}
\usepackage{amsmath,amssymb,amstext}
\usepackage{babel}
\usepackage{times}
\usepackage[Bjarne]{fncychap}
\usepackage{sphinx}

\fvset{fontsize=\small}
\usepackage{geometry}

% Include hyperref last.
\usepackage{hyperref}
% Fix anchor placement for figures with captions.
\usepackage{hypcap}% it must be loaded after hyperref.
% Set up styles of URL: it should be placed after hyperref.
\urlstyle{same}

\addto\captionsenglish{\renewcommand{\figurename}{Fig.\@ }}
\makeatletter
\def\fnum@figure{\figurename\thefigure{}}
\makeatother
\addto\captionsenglish{\renewcommand{\tablename}{Table }}
\makeatletter
\def\fnum@table{\tablename\thetable{}}
\makeatother
\addto\captionsenglish{\renewcommand{\literalblockname}{Listing}}

\addto\captionsenglish{\renewcommand{\literalblockcontinuedname}{continued from previous page}}
\addto\captionsenglish{\renewcommand{\literalblockcontinuesname}{continues on next page}}
\addto\captionsenglish{\renewcommand{\sphinxnonalphabeticalgroupname}{Non-alphabetical}}
\addto\captionsenglish{\renewcommand{\sphinxsymbolsname}{Symbols}}
\addto\captionsenglish{\renewcommand{\sphinxnumbersname}{Numbers}}

\addto\extrasenglish{\def\pageautorefname{page}}

\setcounter{tocdepth}{1}



\title{Django Forms Example Documentation}
\date{Feb 15, 2019}
\release{0.1}
\author{John Strickler}
\newcommand{\sphinxlogo}{\vbox{}}
\renewcommand{\releasename}{Release}
\makeindex
\begin{document}

\pagestyle{empty}
\sphinxmaketitle
\pagestyle{plain}
\sphinxtableofcontents
\pagestyle{normal}
\phantomsection\label{\detokenize{index::doc}}


Contents:


\chapter{Install}
\label{\detokenize{install:install}}\label{\detokenize{install::doc}}
Not yet implemented


\chapter{API}
\label{\detokenize{api:api}}\label{\detokenize{api::doc}}
These are the API views

Other verbage here as much as you want and more :-)


\section{Views}
\label{\detokenize{api:module-apiv1.views}}\label{\detokenize{api:views}}\index{apiv1.views (module)@\spxentry{apiv1.views}\spxextra{module}}\index{EnemyDetail (class in apiv1.views)@\spxentry{EnemyDetail}\spxextra{class in apiv1.views}}

\begin{fulllineitems}
\phantomsection\label{\detokenize{api:apiv1.views.EnemyDetail}}\pysiglinewithargsret{\sphinxbfcode{\sphinxupquote{class }}\sphinxcode{\sphinxupquote{apiv1.views.}}\sphinxbfcode{\sphinxupquote{EnemyDetail}}}{\emph{**kwargs}}{}~\index{serializer\_class (apiv1.views.EnemyDetail attribute)@\spxentry{serializer\_class}\spxextra{apiv1.views.EnemyDetail attribute}}

\begin{fulllineitems}
\phantomsection\label{\detokenize{api:apiv1.views.EnemyDetail.serializer_class}}\pysigline{\sphinxbfcode{\sphinxupquote{serializer\_class}}}
alias of {\hyperref[\detokenize{api:apiv1.serializers.EnemySerializer}]{\sphinxcrossref{\sphinxcode{\sphinxupquote{apiv1.serializers.EnemySerializer}}}}}

\end{fulllineitems}


\end{fulllineitems}

\index{EnemyList (class in apiv1.views)@\spxentry{EnemyList}\spxextra{class in apiv1.views}}

\begin{fulllineitems}
\phantomsection\label{\detokenize{api:apiv1.views.EnemyList}}\pysiglinewithargsret{\sphinxbfcode{\sphinxupquote{class }}\sphinxcode{\sphinxupquote{apiv1.views.}}\sphinxbfcode{\sphinxupquote{EnemyList}}}{\emph{**kwargs}}{}~\index{serializer\_class (apiv1.views.EnemyList attribute)@\spxentry{serializer\_class}\spxextra{apiv1.views.EnemyList attribute}}

\begin{fulllineitems}
\phantomsection\label{\detokenize{api:apiv1.views.EnemyList.serializer_class}}\pysigline{\sphinxbfcode{\sphinxupquote{serializer\_class}}}
alias of {\hyperref[\detokenize{api:apiv1.serializers.EnemySerializer}]{\sphinxcrossref{\sphinxcode{\sphinxupquote{apiv1.serializers.EnemySerializer}}}}}

\end{fulllineitems}


\end{fulllineitems}



\section{Serializers}
\label{\detokenize{api:module-apiv1.serializers}}\label{\detokenize{api:serializers}}\index{apiv1.serializers (module)@\spxentry{apiv1.serializers}\spxextra{module}}\index{CitySerializer (class in apiv1.serializers)@\spxentry{CitySerializer}\spxextra{class in apiv1.serializers}}

\begin{fulllineitems}
\phantomsection\label{\detokenize{api:apiv1.serializers.CitySerializer}}\pysiglinewithargsret{\sphinxbfcode{\sphinxupquote{class }}\sphinxcode{\sphinxupquote{apiv1.serializers.}}\sphinxbfcode{\sphinxupquote{CitySerializer}}}{\emph{instance=None}, \emph{data=\textless{}class 'rest\_framework.fields.empty'\textgreater{}}, \emph{**kwargs}}{}
\end{fulllineitems}

\index{EnemySerializer (class in apiv1.serializers)@\spxentry{EnemySerializer}\spxextra{class in apiv1.serializers}}

\begin{fulllineitems}
\phantomsection\label{\detokenize{api:apiv1.serializers.EnemySerializer}}\pysiglinewithargsret{\sphinxbfcode{\sphinxupquote{class }}\sphinxcode{\sphinxupquote{apiv1.serializers.}}\sphinxbfcode{\sphinxupquote{EnemySerializer}}}{\emph{instance=None}, \emph{data=\textless{}class 'rest\_framework.fields.empty'\textgreater{}}, \emph{**kwargs}}{}
\end{fulllineitems}

\index{PowerSerializer (class in apiv1.serializers)@\spxentry{PowerSerializer}\spxextra{class in apiv1.serializers}}

\begin{fulllineitems}
\phantomsection\label{\detokenize{api:apiv1.serializers.PowerSerializer}}\pysiglinewithargsret{\sphinxbfcode{\sphinxupquote{class }}\sphinxcode{\sphinxupquote{apiv1.serializers.}}\sphinxbfcode{\sphinxupquote{PowerSerializer}}}{\emph{instance=None}, \emph{data=\textless{}class 'rest\_framework.fields.empty'\textgreater{}}, \emph{**kwargs}}{}
\end{fulllineitems}

\index{SuperheroSerializer (class in apiv1.serializers)@\spxentry{SuperheroSerializer}\spxextra{class in apiv1.serializers}}

\begin{fulllineitems}
\phantomsection\label{\detokenize{api:apiv1.serializers.SuperheroSerializer}}\pysiglinewithargsret{\sphinxbfcode{\sphinxupquote{class }}\sphinxcode{\sphinxupquote{apiv1.serializers.}}\sphinxbfcode{\sphinxupquote{SuperheroSerializer}}}{\emph{instance=None}, \emph{data=\textless{}class 'rest\_framework.fields.empty'\textgreater{}}, \emph{**kwargs}}{}
\end{fulllineitems}



\chapter{Models}
\label{\detokenize{models:models}}\label{\detokenize{models::doc}}
Database descriptions

\phantomsection\label{\detokenize{models:module-superheroes.models}}\index{superheroes.models (module)@\spxentry{superheroes.models}\spxextra{module}}
Models for the Superhero database. Contains all data necessary for the API
\index{City (class in superheroes.models)@\spxentry{City}\spxextra{class in superheroes.models}}

\begin{fulllineitems}
\phantomsection\label{\detokenize{models:superheroes.models.City}}\pysiglinewithargsret{\sphinxbfcode{\sphinxupquote{class }}\sphinxcode{\sphinxupquote{superheroes.models.}}\sphinxbfcode{\sphinxupquote{City}}}{\emph{*args}, \emph{**kwargs}}{}
One city. This is the main base city of the superhero.
\index{City.DoesNotExist@\spxentry{City.DoesNotExist}}

\begin{fulllineitems}
\phantomsection\label{\detokenize{models:superheroes.models.City.DoesNotExist}}\pysigline{\sphinxbfcode{\sphinxupquote{exception }}\sphinxbfcode{\sphinxupquote{DoesNotExist}}}
\end{fulllineitems}

\index{City.MultipleObjectsReturned@\spxentry{City.MultipleObjectsReturned}}

\begin{fulllineitems}
\phantomsection\label{\detokenize{models:superheroes.models.City.MultipleObjectsReturned}}\pysigline{\sphinxbfcode{\sphinxupquote{exception }}\sphinxbfcode{\sphinxupquote{MultipleObjectsReturned}}}
\end{fulllineitems}


\end{fulllineitems}

\index{DefaultDate (class in superheroes.models)@\spxentry{DefaultDate}\spxextra{class in superheroes.models}}

\begin{fulllineitems}
\phantomsection\label{\detokenize{models:superheroes.models.DefaultDate}}\pysiglinewithargsret{\sphinxbfcode{\sphinxupquote{class }}\sphinxcode{\sphinxupquote{superheroes.models.}}\sphinxbfcode{\sphinxupquote{DefaultDate}}}{\emph{verbose\_name=None}, \emph{name=None}, \emph{auto\_now=False}, \emph{auto\_now\_add=False}, \emph{**kwargs}}{}~\index{from\_db\_value() (superheroes.models.DefaultDate method)@\spxentry{from\_db\_value()}\spxextra{superheroes.models.DefaultDate method}}

\begin{fulllineitems}
\phantomsection\label{\detokenize{models:superheroes.models.DefaultDate.from_db_value}}\pysiglinewithargsret{\sphinxbfcode{\sphinxupquote{from\_db\_value}}}{\emph{value}, \emph{expression}, \emph{connection}, \emph{context}}{}
Overwrite predefined name from\_db\_value
\begin{quote}\begin{description}
\item[{Parameters}] \leavevmode\begin{itemize}
\item {} 
\sphinxstyleliteralstrong{\sphinxupquote{value}} \textendash{} 

\item {} 
\sphinxstyleliteralstrong{\sphinxupquote{expression}} \textendash{} 

\item {} 
\sphinxstyleliteralstrong{\sphinxupquote{connection}} \textendash{} 

\item {} 
\sphinxstyleliteralstrong{\sphinxupquote{context}} \textendash{} 

\end{itemize}

\item[{Returns}] \leavevmode


\end{description}\end{quote}

\end{fulllineitems}


\end{fulllineitems}

\index{Enemy (class in superheroes.models)@\spxentry{Enemy}\spxextra{class in superheroes.models}}

\begin{fulllineitems}
\phantomsection\label{\detokenize{models:superheroes.models.Enemy}}\pysiglinewithargsret{\sphinxbfcode{\sphinxupquote{class }}\sphinxcode{\sphinxupquote{superheroes.models.}}\sphinxbfcode{\sphinxupquote{Enemy}}}{\emph{*args}, \emph{**kwargs}}{}
An enemy of a Superhero \textendash{} typically a supervillain.
\index{Enemy.DoesNotExist@\spxentry{Enemy.DoesNotExist}}

\begin{fulllineitems}
\phantomsection\label{\detokenize{models:superheroes.models.Enemy.DoesNotExist}}\pysigline{\sphinxbfcode{\sphinxupquote{exception }}\sphinxbfcode{\sphinxupquote{DoesNotExist}}}
\end{fulllineitems}

\index{Enemy.MultipleObjectsReturned@\spxentry{Enemy.MultipleObjectsReturned}}

\begin{fulllineitems}
\phantomsection\label{\detokenize{models:superheroes.models.Enemy.MultipleObjectsReturned}}\pysigline{\sphinxbfcode{\sphinxupquote{exception }}\sphinxbfcode{\sphinxupquote{MultipleObjectsReturned}}}
\end{fulllineitems}


\end{fulllineitems}

\index{Power (class in superheroes.models)@\spxentry{Power}\spxextra{class in superheroes.models}}

\begin{fulllineitems}
\phantomsection\label{\detokenize{models:superheroes.models.Power}}\pysiglinewithargsret{\sphinxbfcode{\sphinxupquote{class }}\sphinxcode{\sphinxupquote{superheroes.models.}}\sphinxbfcode{\sphinxupquote{Power}}}{\emph{*args}, \emph{**kwargs}}{}
A superpower. May be attached to either a Superhero or an Enemy. Name should be short and
distinctive. Description can be verbose.
\index{Power.DoesNotExist@\spxentry{Power.DoesNotExist}}

\begin{fulllineitems}
\phantomsection\label{\detokenize{models:superheroes.models.Power.DoesNotExist}}\pysigline{\sphinxbfcode{\sphinxupquote{exception }}\sphinxbfcode{\sphinxupquote{DoesNotExist}}}
\end{fulllineitems}

\index{Power.MultipleObjectsReturned@\spxentry{Power.MultipleObjectsReturned}}

\begin{fulllineitems}
\phantomsection\label{\detokenize{models:superheroes.models.Power.MultipleObjectsReturned}}\pysigline{\sphinxbfcode{\sphinxupquote{exception }}\sphinxbfcode{\sphinxupquote{MultipleObjectsReturned}}}
\end{fulllineitems}


\end{fulllineitems}

\index{Superhero (class in superheroes.models)@\spxentry{Superhero}\spxextra{class in superheroes.models}}

\begin{fulllineitems}
\phantomsection\label{\detokenize{models:superheroes.models.Superhero}}\pysiglinewithargsret{\sphinxbfcode{\sphinxupquote{class }}\sphinxcode{\sphinxupquote{superheroes.models.}}\sphinxbfcode{\sphinxupquote{Superhero}}}{\emph{*args}, \emph{**kwargs}}{}
A hero with super powers. Like the rest of us, but better in every way.
\index{Superhero.DoesNotExist@\spxentry{Superhero.DoesNotExist}}

\begin{fulllineitems}
\phantomsection\label{\detokenize{models:superheroes.models.Superhero.DoesNotExist}}\pysigline{\sphinxbfcode{\sphinxupquote{exception }}\sphinxbfcode{\sphinxupquote{DoesNotExist}}}
\end{fulllineitems}

\index{Superhero.MultipleObjectsReturned@\spxentry{Superhero.MultipleObjectsReturned}}

\begin{fulllineitems}
\phantomsection\label{\detokenize{models:superheroes.models.Superhero.MultipleObjectsReturned}}\pysigline{\sphinxbfcode{\sphinxupquote{exception }}\sphinxbfcode{\sphinxupquote{MultipleObjectsReturned}}}
\end{fulllineitems}


\end{fulllineitems}



\chapter{Deploy}
\label{\detokenize{deploy:deploy}}\label{\detokenize{deploy::doc}}
This is where you describe how the project is deployed in production.


\chapter{Indices and tables}
\label{\detokenize{index:indices-and-tables}}\begin{itemize}
\item {} 
\DUrole{xref,std,std-ref}{genindex}

\item {} 
\DUrole{xref,std,std-ref}{modindex}

\item {} 
\DUrole{xref,std,std-ref}{search}

\end{itemize}


\renewcommand{\indexname}{Python Module Index}
\begin{sphinxtheindex}
\let\bigletter\sphinxstyleindexlettergroup
\bigletter{a}
\item\relax\sphinxstyleindexentry{apiv1.serializers}\sphinxstyleindexpageref{api:\detokenize{module-apiv1.serializers}}
\item\relax\sphinxstyleindexentry{apiv1.views}\sphinxstyleindexpageref{api:\detokenize{module-apiv1.views}}
\indexspace
\bigletter{s}
\item\relax\sphinxstyleindexentry{superheroes.models}\sphinxstyleindexpageref{models:\detokenize{module-superheroes.models}}
\end{sphinxtheindex}

\renewcommand{\indexname}{Index}
\printindex
\end{document}